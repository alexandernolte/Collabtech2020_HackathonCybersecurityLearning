
Learning Objectives

($IG1$) Participants should be able to develop a deeper understanding of the problem, ($IG2$) Participants should be able to propose ideas that address these security problems, ($IG3$) Participants should be able to communicate the impact that a proposed solution could have on security, ($IG4$) Participant should be able to envision the end product of the solution. 

($ST1$) Participants should have understanding of basic security concepts, ($ST2$) Participants should be able to understand existing security issues, ($ST3$) Participants should be able to understand the impact of security issues.

($MF1$) Participants should be able to incorporate mentor feedback to security solution, ($MF2$) Participant should be able to communicate the process of building the security solution.
 
 ($CS1$) Participants should be able to create a security prototype within the given time frame, ($CS2$) Participant should be able to create a solution prototype that is unique.
 
 \begin{table}[h]
    \centering
    \caption{Learning outcomes to blooms process category}
    \label{tab:learningoutcomes}
    \begin{tabular}{|p{0.2\linewidth}|p{0.2\linewidth}|} \hline
    Learning Outcomes & Process category\\ \hline
    ($IG1$) Participants should have an understanding of current security issues & Understand \\ \hline
    ($IG2$) Participants should be able to propose ideas based on the knowledge gained, to address these security problems & Apply\\ \hline
    ($IG3$) Participants should be able to communicate the impact that the security knowledge gained has on the development of the proposed solution & Analyse \\ \hline
    %($IG4$) Participant should be able to envision the end product of the solution & Analyse \\ \hline
    ($ST1$) Participants should have an understanding of basic security concepts & Understand \\ \hline
    ($ST2$) Participants should be able to understand existing security issues and its impact & Understand \\ \hline
    %($ST3$) Participants should be able to understand the impact of security issues & Understand \\ \hline
    ($MF1$) Participants should be able to incorporate mentor feedback to security solution & Apply \\ \hline
    ($MF2$) Participant should be able to communicate the process of building the security solution as a result of knowledge gained & Analyse \\ \hline
    ($CS1$) Participants should be able to apply the security knowledge gained to create a unique security prototype within the given time frame & Apply \\ \hline
    %($CS2$) Participant should be able to create a solution prototype that is unique & Apply \\ \hline
        \end{tabular}
\end{table}